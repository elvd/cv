%% start of file `template.tex'.
%% Copyright 2006-2013 Xavier Danaux (xdanaux@gmail.com).
%
% This work may be distributed and/or modified under the
% conditions of the LaTeX Project Public License version 1.3c,
% available at http://www.latex-project.org/lppl/.


\documentclass[11pt,a4paper,sans]{moderncv}        % possible options include font size ('10pt', '11pt' and '12pt'), paper size ('a4paper', 'letterpaper', 'a5paper', 'legalpaper', 'executivepaper' and 'landscape') and font family ('sans' and 'roman')
\usepackage{multicol}

% moderncv themes
\moderncvstyle{banking}                             % style options are 'casual' (default), 'classic', 'oldstyle' and 'banking'
\moderncvcolor{orange}                               % color options 'blue' (default), 'orange', 'green', 'red', 'purple', 'grey' and 'black'
%\renewcommand{\familydefault}{\rmdefault}         % to set the default font; use '\sfdefault' for the default sans serif font, '\rmdefault' for the default roman one, or any tex font name
%\nopagenumbers{}                                  % uncomment to suppress automatic page numbering for CVs longer than one page

% adjust the page margins
\usepackage[scale=0.8]{geometry}
%\setlength{\hintscolumnwidth}{3cm}                % if you want to change the width of the column with the dates
%\setlength{\makecvtitlenamewidth}{10cm}           % for the 'classic' style, if you want to force the width allocated to your name and avoid line breaks. be careful though, the length is normally calculated to avoid any overlap with your personal info; use this at your own typographical risks...

% personal data
\name{Viktor}{Doychinov}
%\title{Resumé title}                               % optional, remove / comment the line if not wanted
\address{22 Burchett Terrace}{Leeds}{LS6 2LR}% optional, remove / comment the line if not wanted; the "postcode city" and "country" arguments can be omitted or provided empty
\phone[mobile]{07961~388~940}                   % optional, remove / comment the line if not wanted; the optional "type" of the phone can be "mobile" (default), "fixed" or "fax"
%\phone[fixed]{+2~(345)~678~901}
%\phone[fax]{+3~(456)~789~012}
\email{v.o.doychinov@leeds.ac.uk}                               % optional, remove / comment the line if not wanted
\social[linkedin]{vdoychinov}                        % optional, remove / comment the line if not wanted
%\homepage{about.me/vdoychinov}                         % optional, remove / comment the line if not wanted
%\social[twitter]{jdoe}                             % optional, remove / comment the line if not wanted
\social[github]{elvd}                              % optional, remove / comment the line if not wanted
%\extrainfo{additional information}                 % optional, remove / comment the line if not wanted
\photo[64pt][0.4pt]{hog}                       % optional, remove / comment the line if not wanted; '64pt' is the height the picture must be resized to, 0.4pt is the thickness of the frame around it (put it to 0pt for no frame) and 'picture' is the name of the picture file
%\quote{Some quote}                                 % optional, remove / comment the line if not wanted

\moderncvicons{awesome}
\renewcommand*{\linkedinsocialsymbol}{\faLinkedinSign~~}       % alternative: \faLinkedin
\def\faLinkedinSign{{\FA\symbol{"F08C}}}


%----------------------------------------------------------------------------------
%            content
%----------------------------------------------------------------------------------
\begin{document}
\setlength{\multicolsep}{0pt} 
%-----       resume       ---------------------------------------------------------
\makecvtitle
\linespread{1.1}
\vspace{-1.5em}

\section{Work Experience}
\cventry{}{University of Leeds}{Postdoctoral Research Fellow}{Oct 2015--Sep 2016}{UK}{I work on a project titled "THz Backward Wave Oscillator for Plasma Diagnostics in Nuclear Fusion", in collaboration with researchers at Lancaster University. I am responsible for developing and setting up a fabrication process for rapid realisation and measurement of slow waveguide structures (SWS). In order to do this, I work in a cleanroom environment, using SU-8 photoresist and photolithography methods to implement the different SWS designs. I also participate in their EM evaluation using Ansoft HFSS and am responsible for designing the waveguide holding structures. I am further involved in the `cold' S-parameter measurements using Vector Network Analyzers. Other duties include communication with collaborators, academics, and technical support staff.}
\cventry{also Oct 2014--Nov 2014}{University of Leeds}{Module Demonstrator}{Oct 2015--Nov 2015}{UK}{I assisted the Module Leader of the Year 3 Module "RF and Microwave Engineering" at the School of Electronic and Electrical Engineering during example and computer laboratory classes. I provided one-to-one help to students, explaining concepts and methods to solve problems. I supported students when they were learning how to use the CAD software package Microwave Office, needed for their course projects.}
\cventry{}{Telepoint LTD, Sofia, Bulgaria}{Data Centre Technician}{Jul 2010--Sep 2010}{}{As part of a Network Operations Centre team my duties involved day-to-day maintenance of a data centre, cable installation and testing, mounting and commissioning telecommunication equipment. Additionally, I worked with client technical support to diagnose and troubleshoot various issues.}
\cventry{}{Cosmo Bulgaria Mobile, Sofia}{Microwave Design Intern}{Jul 2008--Aug 2008}{}{As an intern in the Microwave Design Department of a mobile phone operator my main responsibility was design and documentation of microwave frequency point-to-point telecommunication links. Additional responsibilities included communication with Network Traffic Planning Department to make sure requirements are well understood and met.}
\cventry{}{Bulgarian Telecommunications Company, Sofia}{Process Management \& Quality Assurance Intern}{Jul 2007--Oct 2007}{}{During my summer internship as part of the PM \& QA department I participated in the drafting and completion of flowcharts and diagrams representing the business processes of the company, developed in accordance to the eTOM framework. I attended work group meetings between Heads of Departments in order to make sure the flowcharts were accurate and true representation of the actual business practices.}

\section{Education}
\cventry{Expected; Thesis submitted August 2015}{University of Leeds}{PhD Electronic and Electrical Engineering}{2011--2015}{}
{\cvitemwithcomment{}{}{
\begin{description}
    \item[Thesis title] \textit{Quantum Barrier Devices for Sub-Millimetre Wave Detection}\\
    The aim of my project was to study the use of a class of quantum barrier devices as a non-linear element in sub-harmonic mixers operating at millimetre and sub-millimetre frequencies. The devices were further investigated for their application in amplifiers at the same frequency ranges. To achieve this I performed circuit modelling and simulations using Agilent ADS and Ansoft HFSS, in addition to semiconductor device modelling using Matlab and Python. I also fabricated devices and planar circuits in a cleanroom environment, and designed waveveguide blocks and circuits in SolidWorks for manufacture by conventional mechanical means. Throughout the entire time I worked in close collaboration with academics, technical support staff and fellow research students, to ensure smooth and correct work flow. 
\end{description}
}}
\cvitemwithcomment[0.25em]{}{}{
\begin{description}
    \item[\small{Skills gained}] \small{I strengthened my analytical and modelling skills as a result of my project, and improved my numerical skills. I had opportunities to practise my presentation and communication skills via attending research conferences and networking with other researchers. By approaching the PhD from a project management perspective, I gained good task and time management skills, never missing a deadline. I kept myself up-to-date with current developments in my field through webinars and subscriptions to journals. I made sure to attend training courses in measurements offered by equipment vendors, and managed to become a Certified LabVIEW Associate Developer. Furthermore, through training courses at the University of Leeds and elsewhere I improved my Python and FORTRAN knowledge and applied that to my project in order to streamline and speed up data processing.}
\end{description}
} % arguments 3 to 6 can be left empty

\cventry{Awarded with Merit}{University College London}{MSc Spacecraft Technology \& Satellite Communications}{2010--2011}{}{\cvitemwithcomment[0.25em]{}{}{
\begin{description}
    \item[Individual Project] \textit{Alfv\'{e}n -- Magnetosphere-Ionosphere Connection Explorers}\\
    I performed a detailed design of the communication and power subsystems for the Alfven-MICE mission, and investigated possibilities for the Attitude Control Subsystem, as well as an alternative mission scenario. To achieve this, I had to thoroughly research previous missions and use the specialised mission design software package STK. I used Matlab for data processing and link budget design for the communication subsystem. \textit{Project Grade: 65}
    \item[Group Project] \textit{Exploring the Early Solar System: Chariklo Fly-by Mission Concept}
        \begin{itemize}
            \item Appointed Team Leader for Flight Dynamics group
            \item Tasked with designing a suitable trajectory for spacecraft mission under specified restrictions
            \item Communicated with other Team Leaders and Project Supervisors
            \item Successfully presented and argued proposed solution at group project viva
        \end{itemize}
    \item[Key Modules] \hfill
        \begin{itemize}
            \begin{multicols}{2}                    
                \item Space Systems Engineering
                \item Space Design - Electronic Subsystems
                \item Space Instrumentation \& Application
                \item Satellite Communications
                \item RF Circuits \& Subsystems
                \item Communications Systems Modelling
            \end{multicols}
        \end{itemize}
\end{description}
}}

\vspace{0.25em}
\cventry{Overall Grade: 5.3/6}{Technical University--Sofia}{BEng Telecommunications}{2005--2009}{}{\cvitemwithcomment[0.25em]{}{}{
\begin{description}
    \item[Final Year Project] \textit{Vehicular traffic state estimation with Gaussian mixture models}
        \begin{itemize}
            \item Project carried out at Lancaster University as an Erasmus Student
            \item Work consisted of using Matlab to evaluate the performance of Gaussian Mixture Models and their application to the traffic estimation problem
            \item Results of the project were presented at the INFORMATIK 2009 conference
        \end{itemize}
    \item[Other Course Projects] \hfill
        \begin{itemize}
            \item Design of a parabolic antenna with a feed horn for microwave point-to-point link, using Ansoft HFSS
            \item Design of a solid-state microwave amplifier operating at 2.45 GHz, using Ansoft Designer
        \end{itemize}
    \item[Key Modules] \hfill
        \begin{itemize}
            \begin{multicols}{2}                    
                \item Mobile Communications
                \item RF Communications
                \item Microwave Circuits \& Devices
                \item Semiconductor Devices
                \item Antenna and Microwave Technology
                \item Measurement Techniques in Telecommunications
            \end{multicols}
        \end{itemize}
\end{description}}}
%\cventry{Overall Grade: 5.3/6}{Technical University--Sofia}{BEng Telecommunications, Contd.}{2005--2009}{}{\cvitemwithcomment[0.25em]{}{}{
%\begin{description}        
%    \item[Other Course Projects] \hfill
%        \begin{itemize}
%            \item Design of a parabolic antenna with a feed horn for microwave point-to-point link, using Ansoft HFSS
%            \item Design of a solid-state microwave amplifier operating at 2.45 GHz, using Ansoft Designer
%        \end{itemize}
%    \item[Key Modules] \hfill
%        \begin{itemize}
%            \begin{multicols}{2}                    
%                \item Mobile Communications
%                \item RF Communications
%                \item Microwave Circuits \& Devices
%                \item Semiconductor Devices
%                \item Antenna and Microwave Technology
%                \item Measurement Techniques in Telecommunications
%            \end{multicols}
%        \end{itemize}
%\end{description}}
%}

\section{Additional Skills and Competencies}
\cvitem{Specialised Software}{Proficient user of Keysight ADS, Ansoft HFSS, NI LabVIEW, SolidWorks, OriginPro, Matlab.}
\cvitem{OS \& Programming}{Windows 7, Linux, LaTeX, Python, C, HTML/CSS, JavaScript, FORTRAN}
\cvitem{Specialised Equipment}{Oscilloscopes, Network Analyzers, Spectrum Analyzers, Signal Generators, Multimeters}
\cvitem{Technical Skills}{Soldering, Microwave Measurements, Cleanroom Fabrication}

\section{Certificates and Awards}
\cvlistitem{\textbf{Certified LabVIEW Associate Developer}, Licence 100-314-277}
\cvlistitem{\textbf{European Computer Driving Licence}, Licence BCS101808401}
\cvlistitem{\textbf{LFS101x.2: Introduction to Linux}, issued by \textit{\href{https://verify.edx.org/cert/8631080973134729a7f869dd1c54ae7e}{edX.org}}}
\cvlistitem{\textbf{Recipient of an EPSRC Doctoral Training Grant}}
%\cvlistitem{\textbf{UCL Enterprise Boot Camp Participation Certificate}}
\cvlistitem{\textbf{Certificate in Basic Statistics for Researchers}}

\section{Interests and Activities}
\cvitem{CodeClub UK}{I planned, started, and ran a Code Club at a local primary school, successfully completing the first term of coding projects with a group of 12 children. I am currently running Code Club taster sessions at Leeds Central Library, with the aim of establishing a regular club in 2016. I also delivered an example session to staff during their Staff Development Day in November 2015.}
\cvitem{LUU Sci-Fi \& Fantasy Society}{Served as \textbf{Secretary} from March 2013 to March 2014. My responsibilities included ensuring the smooth running of the society, organising regular and one-off events, maintaining files and documentation, communicating with the Student Union and making sure the rest of the Committee Members were kept up-to-date with everything.}
\cvitem{LUU Engineers Without Borders--UK}{Elected \textbf{IT Coordinator} during the inaugural year of the branch, from September 2012 to September 2013. I developed the first iteration of the society's website, and liaised with the EWB's national IT team. I helped with developing the branch by participating in outreach events.}
\cvitem{DIY Electronics}{Arduino and Raspberry Pi enthusiast. Made an African Pygmy Hedgehog activity monitoring system, and continue to improve it.}
\cvitem{Other Interests}{Science and Technology, Board and Card Games, Scientific Computing, Origami, Model Making, World History}

\section{Languages}
\cvlanguage{Bulgarian}{Native}{}
\cvlanguage{English}{Fluent}{\textit{IELTS Score: 8.5 (2009), CAE Grade: A (2003)}}

%\begin{cvcolumns}
%  \cvcolumn{Category 1}{\begin{itemize}\item Person 1\item Person 2\item Person 3\end{itemize}}
%  \cvcolumn{Category 2}{Amongst others:\begin{itemize}\item Person 1, and\item Person 2\end{itemize}(more upon request)}
%  \cvcolumn[0.5]{All the rest \& some more}{\textit{That} person, and \textbf{those} also (all available upon request).}
%\end{cvcolumns}

% Publications from a BibTeX file without multibib
%  for numerical labels: \renewcommand{\bibliographyitemlabel}{\@biblabel{\arabic{enumiv}}}% CONSIDER MERGING WITH PREAMBLE PART
%  to redefine the heading string ("Publications"): \renewcommand{\refname}{Articles}
%\clearpage
\renewcommand{\refname}{Conference Publications}
\nocite{*}
\bibliographystyle{ieeetr}
\bibliography{publications}                        % 'publications' is the name of a BibTeX file

\section{References}
\textit{Available upon request}

% Publications from a BibTeX file using the multibib package
%\section{Publications}
%\nocitebook{book1,book2}
%\bibliographystylebook{plain}
%\bibliographybook{publications}                   % 'publications' is the name of a BibTeX file
%\nocitemisc{misc1,misc2,misc3}
%\bibliographystylemisc{plain}
%\bibliographymisc{publications}                   % 'publications' is the name of a BibTeX file
%
%\clearpage
%%
%%%
%%%-----       letter       ---------------------------------------------------------
%%% recipient data
%\recipient{Institute of Microwaves and Photonics}{School of Electronic and Electrical Engineering\\University of Leeds\\Woodhouse Lane\\Leeds LS2 9JT}
%\date{September 04, 2015}
%\opening{Dear Members of the Search Committee,}
%\closing{Yours sincerely,}
%%\enclosure[Attached]{curriculum vit\ae{}}          % use an optional argument to use a string other than "Enclosure", or redefine \enclname
%\makelettertitle
%I am writing to you in order to apply for the recently advertised in the Institute of Microwaves and Photonics position of a Research Fellow in the Fabrication and Measurement of Submillimetre Wave Backward Wave Oscillator Cavities. I am a PhD student in Microwave Engineering at the IMP, who has recently submitted his thesis, titled `Quantum Barrier Devices for Sub-Millimetre Wave Detection', and am currently awaiting oral examination. 
%
%I firmly believe that my knowledge and expertise in the field of high frequency electronic engineering, combined with my skills in microwave circuit and component design and characterisation, using CAD tools such as Ansoft HFSS and Agilent ADS and modern measurement techniques, would make me an excellent candidate for this position.
%
%My long-standing professional interest has been in microwave engineering, pursuing degrees in the field at both undergraduate and postgraduate level. Having designed and worked with waveguide circuits and components during my studies, I have developed a deep understanding of their application and potential benefits to various other areas, such as atmospheric science, radio astronomy, and short-range wireless communications.
%
%During my PhD project, I have extensively used, and as a result become proficient in, various software tools for microwave circuit design, modelling, and validation. I have used the 3D FEM simulator Ansoft HFSS for applications such as antennas, planar mixer and amplifier circuits, and waveguide couplers and filters. I have also used Agilent ADS for semiconductor device modelling and linear and non-linear circuit design and simulation. Towards the end of my PhD studies, I was also involved in waveguide block design for manufacture, using SolidWorks, for the purposes of investigating sub-millimetre wave finline circuits.
%
%I have complemented this work with vendor-provided training on waveguide and planar circuit measurements, using a wide range of equipment, such as signal generators, vector network analysers, oscilloscopes, spectrum analysers, and power metres. Furthermore, I have developed my skills and knowledge in the area of measurements by becoming a LabVIEW Certified Associate Developer, and applying my LabVIEW skills to semiconductor device measurements for the needs of my project.
%
%Being part of a vibrant Institute with established research teams, I was able to participate and contribute to discussions between peers and partners on a variety of topics. I was also given the opportunity to assist in both undergraduate and MSc students' project supervision, something I thoroughly enjoyed and would be happy to do again.
%
%I am really excited at the possibility to conduct research at the forefront of sub-millimetre wave engineering, and given the opportunity, I am confident that I will be able to contribute successfully to this project, while at the same time further developing my knowledge and skills, and applying them to future research endeavours. 
%
%Thank you very much for your time and consideration. I look forward to hearing from you.
%
%\makeletterclosing
%
%%%%\clearpage\end{CJK*}                              % if you are typesetting your resume in Chinese using CJK; the \clearpage is required for fancyhdr to work correctly with CJK, though it kills the page numbering by making \lastpage undefined
\end{document}

%% end of file `template.tex'.
